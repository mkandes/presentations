\documentclass{beamer}

\usepackage{centernot}
\usepackage{amsmath}
\usepackage{amssymb}
\usepackage{graphics}
\usepackage{hyperref}
\usepackage{setspace}

\title{From Zero to RAPIDS: \\
       Accelerating Data Science and Machine Learning \\
       Workflows on NVIDIA GPUs}
\author{Marty Kandes, Ph.D. \\
   Computational \& Data Science Research Specialist \\
   High-Performance Computing User Services Group \\ 
   San Diego Supercomputer Center \\
   University of California, San Diego}
\date{XSEDE Webinar \\
   Thursday, January 23rd, 2020 \\
   11:00AM - 12:00PM PT}

\begin{document}
\maketitle

\begin{frame}
   \frametitle{About Me}
   \begin{itemize}
      \setlength\itemsep{1.5em}
      \item \textbf{Work}: HPC User Services Group @ SDSC, Comet
      \item \textbf{Background}: Computational Physics, Applied Math, HPC
      \item \textbf{Disclaimer}: \textit{Not} a Data Science or Machine Learning expert
   \end{itemize}
\end{frame}

\begin{frame}
   A few quick reminders ...
\end{frame}

\begin{frame}
   \frametitle{XSEDE Code of Conduct}
   XSEDE has an external code of conduct for XSEDE sponsored events 
   which represents XSEDE's commitment to providing an inclusive and 
   harassment-free environment in all interactions regardless of gender,
   sexual orientation, disability, physical appearance, race, or 
   religion. The code of conduct extends to all XSEDE-sponsored events,
   services, and interactions.
   \\ \ \\
   \url{https://www.xsede.org/codeofconduct}
\end{frame}

\begin{frame}
   \frametitle{Comet: Serving the Long Tail of Science}
   \vspace{-1.0em}
   \begin{figure}[htbp]
      \includegraphics[width=0.8\textwidth]{images/comet.jpg}
   \end{figure}
   \begin{center}
      End of Production Service in March 2021
   \end{center}
\end{frame}

\begin{frame}
   \frametitle{E X P A N S E: Computing Without Boundaries}
   \vspace{-1.0em}
   \begin{figure}[htbp]
      \includegraphics[width=1.0\textwidth]{images/expanse-banner.jpg}
   \end{figure}
   \begin{center}
      Coming September 2020
   \end{center}
\end{frame}

\begin{frame}
   Okay, let's get started ...
\end{frame}

\begin{frame}
   \frametitle{From Zero to RAPIDS: An Overview}
   \begin{itemize}\setlength\itemsep{1.5em}
      \item A Quick Overview of RAPIDS
      \item Jupyter Notebook Demo
      \item Fannie Mae Single-Family Loan Performance Data
      \item Additional Resources and References
      \item Recent and Upcoming RAPIDS Training Events
   \end{itemize}
\end{frame}

\begin{frame}
   \frametitle{What is RAPIDS?}
   \begin{figure}[htbp]
      \includegraphics[width=1.0\textwidth]{images/rapids-logo.png}
   \end{figure}
   RAPIDS is NVIDIA's new suite of open source software libraries and
   application programming interfaces (APIs) that aim to give you the
   ability to accelerate (classical) data science, analytics, and
   machine learning workflows on NVIDIA GPUs.
   \\ \ \\
   \url{https://rapids.ai}
\end{frame}

\begin{frame}
   \frametitle{Why RAPIDS?}
   \begin{itemize}\setlength\itemsep{1.5em}
      \item \textbf{Fast}: Leverages CUDA primitives to provide you with
         out-of-the-box low-level GPU compute optimizations
      \item \textbf{User-Friendly}: Familiar Python interfaces for each
         library allows you to (quickly) integrate GPU parallelism and 
         high-bandwidth memory speeds into your existing workflows
      \item \textbf{Scalable}: Integrates with Dask to allow you to scale-out
         across multiple-node, multi-GPU systems (more easily)
      \item \textbf{Open Source}: Licensed under Apache 2.0
   \end{itemize}
\end{frame}

\begin{frame}
   \frametitle{Traditional (CPU-based) Data Science Software Stack}
   \begin{figure}[htbp]
      \includegraphics[width=1.0\textwidth]{images/traditional-data-science-software-stack.png}
   \end{figure}
\end{frame}

\begin{frame}
   \frametitle{RAPIDS (GPU-based) Data Science Software Stack}
   \begin{figure}[htbp]
      \includegraphics[width=1.0\textwidth]{images/nvidia-rapids-data-science-software-stack.png}
   \end{figure}
\end{frame}

\begin{frame}
   \frametitle{RAPIDS: cuDF}
   \begin{figure}[htbp]
      \includegraphics[width=0.8\textwidth]{images/pandas-logo.png}
   \end{figure}
   \ \\ \ \\
   \begin{itemize}\setlength\itemsep{1.5em}
      \item cuDF is NVIDIA's GPU-accelerated version of Pandas.
      \item It provides you with a GPU-native DataFrame library for 
         loading, joining, aggregating, filtering, and otherwise 
         manipulating data.
      \item It provides you with a familiar Pandas-like Python API that
         aims to easily accelerate your Pandas-based workflows on NVIDIA 
         GPUs without having to understand the details of CUDA programming.
   \end{itemize}
\end{frame}

\begin{frame}
   \frametitle{RAPIDS: cuML}
   \begin{figure}[htbp]
      \includegraphics[width=0.5\textwidth]{images/scikit-learn-logo.png}
   \end{figure}
   \ \\ \ \\
   \begin{itemize}\setlength\itemsep{1.5em}
      \item cuML is NVIDIA's GPU-accelerated version of Scikit-learn.
      \item It provides you with a GPU-acclerated library of standard 
         statistical and (classical) machine learning algorthms: Linear
         Regression, K-Means, SVD, etc.
   \end{itemize}
\end{frame}

\begin{frame}
   \frametitle{RAPIDS: cuGraph}
   \begin{itemize}\setlength\itemsep{1.5em}
      \item cuGraphs is NVIDIA's GPU-acclerated version of NetworkX. 
      \item It provides you with a collection of graph analytics 
         algorithms that can be used to process data found in cuDF 
         DataFrames: Page Rank, Breadth First Search, etc.
      \item Up to 500M edges on a single 32 GB NVIDIA GPU; scales to
         billions of edges on multi-GPUs.
   \end{itemize}
\end{frame}

\begin{frame}
   \frametitle{RAPIDS: cuSpatial}
   \begin{itemize}\setlength\itemsep{1.5em}
      \item cuSpatial is NVIDIA's GPU-acclerated library for geospatial
         and spatiotemporal data processing.
      \item It provides you with a collection of common spatial and 
         spatiotemporal operations: point-in-polygon tests, distances 
         between trajectories, trajectory clustering, etc.
   \end{itemize}
\end{frame}

\begin{frame}
   \frametitle{Jupyter Notebook Demo}
   \begin{figure}[htbp]
      \includegraphics[width=0.7\textwidth]{images/the-big-short.png}
   \end{figure}
\end{frame}

\begin{frame}
   \frametitle{Fannie Mae Single-Family Loan Performance Data}
   \begin{figure}[htbp]
      \includegraphics[width=0.8\textwidth]{images/fannie-mae-sflp-website.png}
   \end{figure}
   \url{https://www.fanniemae.com/portal/funding-the-market/data/loan-performance-data.html}
\end{frame}

\begin{frame}
   \frametitle{Why the Fannie Mae SFLP Dataset?}
   \begin{itemize}\setlength\itemsep{1.5em}
      \item \textbf{Large in Size}: The dataset contains more than 1.9 
         billion records on 37 million 30-year fixed rate mortgages. It
         is approximately 200 GB uncompressed.
      \item \textbf{Growing}: The dataset is updated quarterly by Fannie Mae.
      \item \textbf{Well-documented}: Fannie Mae provides an in-depth tutorial
         on analyzing the dataset with R/SAS code examples. NVIDIA has 
         also featured the dataset in previous RAPIDS demonstrations.
      \item \textbf{Easily accessible}: The dataset can be downloaded for free
         from Fannie Mae. Several subsets are available from NVIDIA on 
         the RAPIDS website. 
   \end{itemize}
\end{frame}

\begin{frame}
   \frametitle{What is Fannie Mae?}
   \vspace{-1.0em}
   \begin{figure}[htbp]
      \includegraphics[width=0.3\textwidth]{images/fannie-mae-logo.png}
   \end{figure}
   \begin{itemize}\setlength\itemsep{1.0em}
      \item Federal National Mortgage Association (FNMA)
      \item Government-Sponsored Enterprise
      \item Founded in 1938
      \item Created the secondary mortgage market in the U.S.
      \item Became a publicly traded company in 1968
      \item Placed into conservatorship by U.S. federal government in 
            2008 along with Freddie Mac
   \end{itemize}
\end{frame}

\begin{frame}
   \frametitle{What does Fannie Mae do?}
   \begin{figure}[htbp]
      \includegraphics[width=1.0\textwidth]{images/how-does-fannie-mae-work.jpg}
   \end{figure}
\end{frame}

\begin{frame}
   \frametitle{Size of U.S. Residential Mortgage Market}
   \begin{figure}[htbp]
      \includegraphics[width=0.8\textwidth]{images/mortgage-market-size.jpg}
   \end{figure}
\end{frame}

\begin{frame}
   \frametitle{Conforming Mortgage Loans}
   \begin{itemize}\setlength\itemsep{1.0em}
      \item Primary type of mortgage loans that Fannie Mae and other 
         GSEs can purchase.
      \item Limits maximum loan amount for a given type of property
      \item In 2019, the standard conforming loan limit for a single 
         family home was \$484,350. 
      \item \textit{Jumbo conforming loan limits}: In 2008, conforming 
         loan limits in high cost areas were raised to \$729,750 or 
         125\% of the median home value within the metropolitan
         statistical area, whichever is the lesser.
   \end{itemize}
\end{frame}

\begin{frame}
   Okay, let's go to the Jupyter Notebook demo ...
   \\ \ \\
   \url{https://github.com/mkandes/presentations/blob/master/2020/01/23/from-zero-to-rapids/notebooks/fannie-mae-sflp.ipynb}
\end{frame}

\begin{frame}
   \frametitle{Additional Resources and References}
   \begin{itemize}\setlength\itemsep{1.5em}
      \item \textbf{Homepage}: \url{https://rapids.ai}
      \item \textbf{Blog}: \url{https://medium.com/rapids-ai}
      \item \textbf{Intro Tutorials}: \url{https://docs.rapids.ai/start}
      \item \textbf{GitHub}: \url{https://github.com/rapidsai}
   \end{itemize}
\end{frame}

\begin{frame}
   \frametitle{Jupyter Notebooks}
   \begin{itemize}\setlength\itemsep{1.5em}
      \item \textbf{NVIDIA Notebooks}: \url{https://github.com/rapidsai/notebooks}
      \item \textbf{Community Contributed Notebooks}: \url{https://github.com/rapidsai/notebooks-contrib}
      \item \textbf{Using RAPIDS + PyTorch on the SFLP Dataset}: \url{https://github.com/rapidsai/notebooks-contrib/tree/branch-0.12/blog_notebooks/mortgage_deep_learning}
   \end{itemize}
\end{frame}

\begin{frame}
   \frametitle{RAPIDS API Documentation}
   \begin{itemize}\setlength\itemsep{1.5em}
      \item \url{https://rapidsai.github.io/projects/cudf/en/0.11.0/api.html}
      \item \url{https://rapidsai.github.io/projects/cuml/en/0.11.0/api.html}
      \item \url{https://rapidsai.github.io/projects/cugraph/en/0.11.0/api.html}
   \end{itemize}
\end{frame}

\begin{frame}
   \frametitle{Recent and Upcoming RAPIDS-Related Training Events}
   \textit{CUDA-Python and RAPIDS for blazing fast scientific computing}
   \\ \ \\
   Abraham Stern, Solutions Architect, NVIDIA 
   \\ \ \\
   \begin{itemize}\setlength\itemsep{1.5em}
      \item \textbf{XSEDE ECSS Symposium} \\
            Tuesday, January 21st, 2020 \\
            \url{https://youtu.be/NdKWEkV9X34}  
      \item \textbf{XSEDE Webinar} \\
            Thursday, February 20, 2020 \\
\url{https://www.sdsc.edu/education_and_training/training/202002_gpu_accelerated_computing_with_cuda_python.html}
   \end{itemize}
\end{frame}

\begin{frame}
   \frametitle{Questions?}
   \begin{figure}[htbp]
      \includegraphics[width=0.6\textwidth]{images/bill-excels.jpg}
   \end{figure}
\end{frame}

\end{document}
